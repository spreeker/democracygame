\documentclass[a4paper]{report}
\usepackage{hyperref}
\usepackage{a4wide}
\author{Thijs Coenen}
\title{Emocracy Technical Documentation march release}
\begin{document}
\maketitle
\tableofcontents
\chapter{Emocracy REST API}
By providing an API Emocracy allows third parties to construct different 
interfaces to the game. The API will allow both other websites and 
software running on desktops or mobile platforms to interact with the game 
(provided that an internet connection is available). Authentication will be 
implemented using the OAuth protocol (see \url{http://oauth.net}).

The API for the game is implemented as a REST (REpresentational State Transfer) 
API. REST was formalized by Roy Fielding in his doctoral thesis (see : 
\url{http://www.ics.uci.edu/\~fielding/pubs/dissertation/top.htm}). REST APIs do
not define an extra protocol layer on top of HTTP (like XML-RPC and other 
webservices protocols do).

\section{Documentation conventions}
In this document \texttt{this typeface} is used for API resource indentifiers 
and parameter names. For variable parts of resource identifiers \emph{this 
typeface}. For each resource that the REST API provides, the documentation lists
the accepted HTTP methods (see \ref{httpmethods}) and the HTTP response codes 
that the API clients can expect to receive (see \ref{httpstatuscodes}).

\section{HTTP methods used by the Emocracy REST API}\label{httpmethods}
In a REST API the HTTP methods are used to request data or change it, there is
no extra protocol layer defined on top of HTTP. The first version of the game
REST API supports only HTTP GET and POST. The former to request data and the 
latter for creating, changing or deleting resources.
\begin{itemize}
    \item{\texttt{GET}\\
    Used to get information out of a resource, this method does not change the 
    state of the Resource.}
    \item{\texttt{POST}\\
    Used to create new resources and to change or delete resources. 
    \emph{Future use will be only the creation of resources.}}
    \item{\texttt{PUT}\\
    \emph{Not currently supported. Future use will be editing resources.}}
    \item{\texttt{DELETE}\\
    \emph{Not currently supported. Future use will be deleting resources.}}
\end{itemize}
For more information see \mbox{\url{http://www.w3.org/Protocols/rfc2616/rfc2616-sec9.html}}
\section{HTTP Status Codes used by Emocracy REST API}\label{httpstatuscodes}
REST APIs use the status codes that the HTTP protocol defines to communicate the
succes or failure of a request. The game API uses a subset of HTTP status
codes listed below:
\begin{itemize}
    \item{\texttt{200, Ok}\\
    Returned when a requested resource was found.
    }
    \item{\texttt{201, Created}\\
    Returned after a new resource was created (after a POST to the correct 
    resource).
    }
    \item{\texttt{400, Bad Request}\\
    Returned when POST data contains bad data or there is a problem processing 
    the request.
    }
    \item{\texttt{401, Unauthorized}\\
    Resource is not available without authorization. With proper authentication
    the resource is available.
    }
    \item{\texttt{403, Forbidden}\\
    Request is refused. Authentication won't help.
    }
    \item{\texttt{404, Not Found}\\
    Resource was not found.
    }
    \item{\texttt{405, Method Not Allowed}\\
    Request method is not allowed on this resource.
    }
    \item{\texttt{500, Internal Server Error}\\
    An error ocurred in the server.
    }
\end{itemize}

Some REST APIs support suppressing the status codes. They always send HTTP 
status 200 and put the actual response status in the response body. This is 
done to allow browser based apps (implemented in Flash or Javascript) to acces
the status codes that a browser normally filters out. See the 
\texttt{suppress\_response\_codes} parameter that the Twitter REST API supports
for more information (\mbox{\url{http://apiwiki.twitter.com/REST+API+Documentation}}).\\\\
See also : \mbox{\url{http://www.w3.org/Protocols/rfc2616/rfc2616-sec10.html}.}


\section{Emocracy REST API data types}
\subsection{Requests}
The first version of the game API sends data as JSON (JavaScript Object Notation)
to API clients. Although JSON is based on a subset of JavaScript it can be used
with other programming languages and offers a lightweight alternative to XML.
See \url{http://www.json.org/} for more information on JSON and pointers to 
libraries implementing JSON. 
\subsection{Responses}
Data send by the client to the game API is encoded as normal POST data. The 
first version of the game API supports only HTTP POST for creating, changing
or deleting resources. Future versions of the API should also support HTTP PUT 
and DELETE, that means that future versions of the API will also have to be able
to receive JSON or XML.

\section{API versioning}
The Emocracy REST API is still evolving therefore it is versioned. This document
describes the design of the first (\texttt{v0}) version of the REST API. The 
implementation is tested internally (within the Emocracy project) by building a
mobile phone web interface that uses the REST API to play the game.

\section{Resources provided by the Emocracy REST API}\label{resources}
This section describes the resources made available trough the game API. As 
of now not all resources have been defined yet.
\subsection{\texttt{http://emocracy.nl/api/v0/issues/}}
\begin{itemize}
    \item{\textsl{implemented by} : \texttt{api.views.IssueCollection}}
    \item{\textsl{accepted http methods} :
        \begin{itemize}
            \item{\texttt{GET} : Get a list of issues.
                \begin{itemize}
                    \item{\texttt{200}
                    \begin{itemize}
                        \item{Request parameters encoded as GET parameters. 
                        \texttt{page} for the page of results. \texttt{sort\_order}
                        for the ordering of the returned issues possible values are
                        \texttt{time\_stamp}, \texttt{votes}, \texttt{score} or \texttt{hotness}.
                        \texttt{perpage} can be set to change the number of 
                        resources that are returned per request.
                        }
                        \item{Response data are a JSON encoded paginated list of 
                        resource URIs (\texttt{resources}) and the URIs of the 
                        previous and next page (\texttt{previous} and \texttt{next}
                        ).}
                    \end{itemize}
                    }
                \end{itemize}
             }
            \item{\texttt{POST} : Create an Issue
                \begin{itemize}
                    \item{\texttt{401}, \texttt{400}}
                    \item{\texttt{201}
                    \begin{itemize}
                        \item{Request parameters encoded as POST data.}
                        \item{Response is a JSON encoded URI of the newly
                        created Issue (\texttt{resource}).}
                    \end{itemize}
                    }
                \end{itemize}
            }
            \item{other : 
                \begin{itemize}
                    \item{\texttt{405}}
                \end{itemize}
            }
        \end{itemize}
    }
\end{itemize}


\subsection{\texttt{http://emocracy.nl/api/v0/issues/\emph{issue\_pk}/}}
\begin{itemize}
    \item{\textsl{implemented by} : \texttt{api.views.IssueResource}}
    \item{\textsl{accepted http methods} :
        \begin{itemize}
            \item{\texttt{GET} : Get detailed info about an Issue with primary key \emph{issue\_pk}.
                \begin{itemize}
                    \item{\texttt{404}}
                    \item{\texttt{200}
                    \begin{itemize}
                        \item{Response data is a JSON encoded Issue.}
                    \end{itemize}
                    }
                \end{itemize}
            }
            \item{\texttt{other}
                \begin{itemize}
                    \item{\texttt{405}}
                \end{itemize}
            }
        \end{itemize}
    }
\end{itemize}


\subsection{\texttt{http://emocracy.nl/api/v0/issues/\emph{issue\_pk}/tags/}}
\begin{itemize}
    \item{\textsl{implemented by} : \texttt{api.views.IssueTagCollection}}
    \item{\textsl{accepted http methods} :
        \begin{itemize}
            \item{\texttt{GET} : 100 most popular tags for issue with primary key \emph{issue\_pk}.
                \begin{itemize}
                    \item{\texttt{404}}
                    \item{\texttt{200}
                        \begin{itemize}
                        \item{Request parameters encoded as GET parameters. 
                            \texttt{page} for the page of results.
                            \texttt{perpage} can be set to change the number of 
                            resources that are returned per request.                        
                        }
                        \item{
                            Requested issue found, returned data are a JSON encoded
                            paginated list of tags (\texttt{resources}) and the URIs of 
                            the previous and next page (\texttt{previous} and \texttt{next}).
                        }
                        \end{itemize}
                    }
                \end{itemize}
            }
            \item{\texttt{POST} : Tag issue with primary key \emph{issue\_pk}.
                \begin{itemize}
                    \item{\texttt{400, 401}}
                    \item{\texttt{201}
                    \begin{itemize}
                    \item{Request data is encoded as POST data. The name of the 
                    tag is encoded as UTF-8 with key \texttt{tagname}.}
                    \item{Tag added to the Issue with primary key \emph{pk}. The 
                    URI of the tag resource is returned JSON encoded.}
                    \end{itemize}
                    }
                \end{itemize}
            }
            \item{\texttt{other}
                \begin{itemize}
                    \item{\texttt{405}}
                \end{itemize}
            }
        \end{itemize}
    }
\end{itemize}

\subsection{\texttt{http://emocracy.nl/api/v0/issues/\emph{issue\_pk}/votes/}}
\begin{itemize}
    \item{\textsl{implemented by} : \texttt{api.views.IssueVoteCollection}}
    \item{\textsl{accepted http methods} :
        \begin{itemize}
            \item{\texttt{GET} : Get the votes for the Issue identified by primary key \emph{issue\_pk}.
                \begin{itemize}
                    \item{\texttt{401, 404}}
                    \item{\texttt{200}
                    \begin{itemize}
                        \item{Request parameters encoded as GET parameters. 
                        \texttt{page} for the page of results.
                        \texttt{perpage} can be set to change the number of 
                        resources that are returned per request.                        
                        }
                        \item{Response data are a JSON encoded paginated list of 
                        resource URIs (\texttt{resources}) and the URIs of the 
                        previous and next page (\texttt{previous} and \texttt{next}
                        ).}
                    \end{itemize}
                    }
                \end{itemize}
            }
            \item{\texttt{POST} : Vote on the Issue identified by primary key \emph{issue\_pk}.
                \begin{itemize}
                    \item{\texttt{400, 401, 404}}
                    \item{\texttt{201}
                        \begin{itemize}
                            \item{Voted on Issue, return data contains the JSON 
                            encoded URI of the vote.}
                        \end{itemize}
                    }
                \end{itemize}
            }
            \item{\texttt{other}
                \begin{itemize}
                    \item{\texttt{405}}
                \end{itemize}
            }
        \end{itemize}
    }
\end{itemize}

\subsection{\texttt{http://emocracy.nl/api/v0/votes/}}
\begin{itemize}
    \item{\textsl{implemented by} : \texttt{api.views.VoteCollection}}
    \item{\textsl{accepted http methods} :
        \begin{itemize}
            \item{\texttt{GET} : Get a list of public votes.
                \begin{itemize}
                    \item{\texttt{200} 
                    \begin{itemize}
                        \item{Request parameters encoded as GET parameters. 
                        \texttt{page} for the page of results.}
                        \item{Response data are a JSON encoded paginated list of 
                        resource URIs (\texttt{resources}) and the URIs of the 
                        previous and next page (\texttt{previous} and \texttt{next}
                        ).}
                    \end{itemize}
                    }
                \end{itemize}
            }
            \item{\texttt{other} :
                \begin{itemize}
                    \item{\texttt{405}}
                \end{itemize}
            }
        \end{itemize}
    }
    \item{\textsl{note} : No HTTP POST is accepted because votes are created
    by POSTing to the\\
    \mbox{\texttt{http://emocracy.nl/api/v0/issues/\emph{issue\_pk}/votes/}}
    resource.
    }
\end{itemize}

\subsection{\texttt{http://emocracy.nl/api/v0/votes/\emph{vote\_pk}/}}
\begin{itemize}
    \item{\textsl{implemented by} : \texttt{api.views.VoteResource}}
    \item{\textsl{accepted http methods} :
        \begin{itemize}
            \item{\texttt{GET} : Get information about a vote with primary key \emph{vote\_pk}. 
                \begin{itemize}
                    \item{\texttt{401, 404}}
                    \item{\texttt{200}
                        \begin{itemize}
                            \item{Response data is a JSON encoded representation of 
                            the vote (date / time, user, issue).}
                        \end{itemize}
                    }
                \end{itemize}
            }
            \item{\texttt{other} :
                \begin{itemize}
                    \item{\texttt{405}}
                \end{itemize}
            }
        \end{itemize}
    }
\end{itemize}

\subsection{\texttt{http://emocracy.nl/api/v0/users/}}
\begin{itemize}
    \item{\textsl{implemented by} : \texttt{api.views.UserCollection}}
    \item{\textsl{accepted http methods} :
        \begin{itemize}
            \item{\texttt{GET} : Get a list of users.
                \begin{itemize}
                    \item{\texttt{200}
                    \begin{itemize}
                        \item{Request parameters encoded as GET parameters. 
                        \texttt{page} for the page of results.}
                        \item{Response data are a JSON encoded paginated list of 
                        resource URIs (\texttt{resources}) and the URIs of the 
                        previous and next page (\texttt{previous} and \texttt{next}
                        ).}
                    \end{itemize}
                    }
                \end{itemize}
            }
            \item{\texttt{other} :
                \begin{itemize}
                    \item{\texttt{405}}
                \end{itemize}
            }
        \end{itemize}
    }
    \item{\textsl{note} : User creation through API not allowed as yet.}
\end{itemize}

\subsection{\texttt{http://emocracy.nl/api/v0/users/\emph{user\_pk}/}}
\begin{itemize}
    \item{\textsl{implemented by} : \texttt{api.views.UserResource}}
    \item{\textsl{accepted http methods} :
        \begin{itemize}
            \item{\texttt{GET} : User info for the user with primary key \emph{user\_pk}.
                \begin{itemize}
                    \item{\texttt{200}
                    \begin{itemize}
                        \item{Response data are JSON representations of a user and
                        his or her profile containing such data as user's score
                        and role within the game.}
                    \end{itemize}
                    }
                \end{itemize}
            }
            \item{\texttt{other} :
                \begin{itemize}
                    \item{\texttt{405}}
                \end{itemize}
            }
        \end{itemize}
    }
\end{itemize}


\subsection{\texttt{http://emocracy.nl/api/v0/tags/}}
\begin{itemize}
    \item{\textsl{implemented by} : \texttt{api.views.TagCollection}}
    \item{\textsl{accepted http methods} :
        \begin{itemize}
            \item{\texttt{GET}: List of tags sorted by popularity.
                \begin{itemize}
                    \item{\texttt{200} 
                    \begin{itemize}
                        \item{Request parameters encoded as GET parameters. 
                        \texttt{page} for the page of results.}
                        \item{Response data are a JSON encoded paginated list of 
                        resource URIs (\texttt{resources}) and the URIs of the 
                        previous and next page (\texttt{previous} and \texttt{next}
                        ).}
                    \end{itemize}
                    }
                \end{itemize}
            }
            
            \item{\texttt{other} :
                \begin{itemize}
                    \item{\texttt{405}}
                \end{itemize}
            }
        \end{itemize}
    }
\end{itemize}


\subsection{\texttt{http://emocracy.nl/api/v0/tags/\emph{tag\_pk}/}}
\begin{itemize}
    \item{\textsl{implemented by} : \texttt{api.views.TagResource}}
    \item{\textsl{accepted http methods} :
        \begin{itemize}
            \item{\texttt{GET} : Information about the tag with primary key \emph{taag\_pk}.
                \begin{itemize}
                    \item{\texttt{200}
                    \begin{itemize}
                        \item{Response data is JSON encoded number of times the
                        tag is used (\texttt{count}) and the tag itself (\texttt{tagname})}
                    \end{itemize}
                    }
                \end{itemize}
            }
            
            \item{\texttt{other} :
                \begin{itemize}
                    \item{\texttt{405}}
                \end{itemize}
            }
        \end{itemize}
    }
\end{itemize}

\subsection{\texttt{http://emocracy.nl/api/v0/tags/\emph{tag\_pk}/issues/}}
\begin{itemize}
    \item{\textsl{implemented by} : \texttt{api.views.TagIssueCollection}}
    \item{\textsl{accepted http methods} :
        \begin{itemize}
            \item{\texttt{GET} : List of Issues tagged with the tag that has primary key \emph{tag\_pk}.}
                \begin{itemize}
                    \item{\texttt{200}
                    \begin{itemize}
                        \item{Request parameters encoded as GET parameters 
                        \texttt{page} for the page of results.}
                        \item{Response data are a JSON encoded paginated list of 
                        resource URIs (\texttt{resources}) and the URIs of the 
                        previous and next page (\texttt{previous} and \texttt{next}
                        ).}
                    \end{itemize}
                    }
                \end{itemize}
            
            
            \item{\texttt{other} :
                \begin{itemize}
                    \item{\texttt{405}}
                \end{itemize}
            }
        \end{itemize}
    }
\end{itemize}



\subsection{\texttt{http://emocracy.nl/api/v0/candidates/}}
In the game 
\begin{itemize}
    \item{\textsl{implemented by} : \texttt{api.views.CandidateCollection}}
    \item{\textsl{accepted http methods} :
        \begin{itemize}
            \item{\texttt{GET}: Get a list of current candidacies.
                \begin{itemize}
                    \item{\texttt{200} 
                    \begin{itemize}
                        \item{Request parameters encoded as GET parameters 
                        \texttt{page} for the page of results.}
                        \item{Response data are a JSON encoded paginated list of 
                        resource URIs (\texttt{resources}) and the URIs of the 
                        previous and next page (\texttt{previous} and \texttt{next}
                        ).}
                    \end{itemize}
                    }
                \end{itemize}
            }
            \item{\texttt{POST} : Let a player run for office.
                \begin{itemize}
                    \item{\texttt{400, 401}}
                    \item{\texttt{201} : Candidate created.
                    \begin{itemize}
                        \item{Request data are a http POST encoded representation
                        of the candidate.}
                        \item{Response data are JSON encoded URI of the new
                        candidate resource that was created.}
                    \end{itemize}
                    }
                \end{itemize}
            }
            
            \item{\texttt{other} :
                \begin{itemize}
                    \item{\texttt{405}}
                \end{itemize}
            }
        \end{itemize}
    }
    \item{\textsl{note} : See the note for section \ref{candidacy} for more information.}    
\end{itemize}


\subsection{\texttt{http://emocracy.nl/api/v0/candidates/\emph{candidate\_pk}/}}
\begin{itemize}
    \item{\textsl{implemented by} : \texttt{api.views.CandidateResource}}
    \item{\textsl{accepted http methods} :
        \begin{itemize}
            \item{\texttt{GET} : Info about a candidacy with primary key \emph{candidate\_pk}.
                \begin{itemize}
                    \item{\texttt{200}
                    \begin{itemize}
                        \item{Response data are JSON encoded, content to be determined.}
                    \end{itemize}
                    }
                \end{itemize}
            }
            \item{\texttt{POST} : End a candidacy.
                \begin{itemize}
                    \item{\texttt{400, 401}}
                    \item{\texttt{201} : Candidacy ended.
                        \begin{itemize}
                            \item{Response data are JSON encoded, content to be determined.}
                        \end{itemize}
                      }
                \end{itemize}
            }
            
            \item{\texttt{other} :
                \begin{itemize}
                    \item{\texttt{405}}
                \end{itemize}
            }
        \end{itemize}
    }
    \item{\textsl{note} : 
    The idea behind the two candidate resources is to provide a list of
    current candidacies and an archive of the previous candidacies. A
    candidate resource is never really deleted it is only marked inactive.
    }
    
\end{itemize}

\subsection{\texttt{http://emocracy.nl/api/v0/polls/}}
\begin{itemize}
    \item{\textsl{implemented by} : \texttt{api.views.PollCollection}}
    \item{\textsl{accepted http methods} :
        \begin{itemize}
            \item{\texttt{GET} : List of polls.
                \begin{itemize}
                    \item{\texttt{200}}
                    \begin{itemize}
                        \item{Request parameters encoded as GET parameters 
                        \texttt{page} for the page of results.}
                        \item{Response data are a JSON encoded paginated list of 
                        resource URIs (\texttt{resources}) and the URIs of the 
                        previous and next page (\texttt{previous} and \texttt{next}
                        ).}
                    \end{itemize}
                \end{itemize}
            }
            \item{\texttt{POST} : Create a poll.
                \begin{itemize}
                    \item{\texttt{401, 400}}
                    \item{\texttt{201}
                    \begin{itemize}
                        \item{Request data are a http POST encoded 
                        representation of a poll.}
                        \item{Response data are a JSON encoded URI for the newly
                        created poll.}
                    \end{itemize}
                    }
                \end{itemize}
            }
            
            \item{\texttt{other} :
                \begin{itemize}
                    \item{\texttt{405}}
                \end{itemize}
            }
        \end{itemize}
    }
\end{itemize}

\subsection{\texttt{http://emocracy.nl/api/v0/polls/\emph{poll\_pk}/}}
\begin{itemize}
    \item{\textsl{implemented by} : \texttt{api.views.PollResource}}
    \item{\textsl{accepted http methods} :
        \begin{itemize}
            \item{\texttt{GET} :
                \begin{itemize}
                    \item{\texttt{200} : info about the poll with primary key \emph{poll\_pk}
                    \begin{itemize}
                        \item{Response data are a JSON encode representation of
                        the Poll also containing a list of URIs to the Issues 
                        that go with the poll.}
                    \end{itemize}                    
                    }
                \end{itemize}
            }
            \item{\texttt{POST} : create a new Poll.
                \begin{itemize}
                    \item{\texttt{400, 401}}
                    \item{\texttt{201}
                        \begin{itemize}
                            \item{Poll created, the response data is a JSON 
                            encoded URI to the newly created Poll.}
                        \end{itemize}
                    }
                \end{itemize}
            }
            
            \item{\texttt{other} :
                \begin{itemize}
                    \item{\texttt{405}}
                \end{itemize}
            }
        \end{itemize}
    }
\end{itemize}

\section{Authentication through OAuth}
The game API needs to allow access to user data. The Emocracy project does not 
want to share log in credentials with third parties. Third parties should also 
identify themselves with Emocracy to be able to act on abuse of the API. OAuth 
is a new standard that is freely implementable for authentication of APIs. 

OAuth comes with some jargon of its own. Users of the API are called consumers, 
they are identified by a consumer key. Request to the API are only allowed if
they are accompanied by an authenticated acces token. To get an acces token the
consumer needs to get a request token that needs to be authorized. OAuth can be 
used in two modes. One is called two legged and the other is called three 
legged. The two legged authentication is used in case there are only an API 
provider and a consumer. The three legged version of authentication through 
OAuth is used when there is an API provider and a consumer that needs acces to
private data stored on the server of the API provider. In that case the users 
whose data is requested need to authorize the consumer before it gets acces.

Emocracy will use three legged authentication because it stores data for its 
users. Consumers need to register with Emocracy to get a consumer key and 
consumer secret. When abuse of the API is detected the consumer key will be 
retracted. 

For more information on OAuth:
\begin{itemize}
    \item{The OAuth website \url{http://oauth.net}}
    \item{The Netflix OAuth documentation \url{http://developer.netflix.com/docs/Security}}
    \item{The Google data API OAuth documentation \url{http://code.google.com/intl/nl-NL/apis/gdata/articles/oauth.html}}
    \item{Yahoo developer information about OAuth \url{http://developer.yahoo.com/oauth/guide/}}
\end{itemize}

\section{Implementation status}
OAuth is not yet hooked up to the API, which means that the API cannot yet allow
acces to protected resources. The exact data returned by the API resource are 
subject to change as practical experience is gained through the use of the API.
Another issue is how much of the API will run over encrypted connections (using
HTTPS) --- an issue that affects the resources needed to run Emocracy.
\section{Test server location}
The Emocracy test server is located at : \url{http://demo.emocracy.nl} . So all
Resources mentioned in section \ref{resources} have as \url{http://demo.emocracy.nl}
as server name in stead of \url{http://emocracy.nl} .

\end{document}